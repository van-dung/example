\documentclass[10pt]{book}

%%%%%%%%%%
% Packages utilises
%%%%%%%%%%
% Langue
\usepackage[T1]{fontenc}
\usepackage[mac]{inputenc}
\usepackage[frenchb]{babel} % On peut aussi mettre english ici.
% Figures et couleurs
\usepackage{graphicx}
\usepackage{subfigure}
\usepackage{color}
\definecolor{gris}{gray}{0.45}
% Font math
\usepackage{amssymb, amsmath, amsfonts}
\usepackage{dsfont}
% Mise en page latex et police
\usepackage{vmargin}

%%%%%%%%%%
\begin{document}
%%%%%%%%%%

%\addtocounter{page}{-1} % Pour revenir a 0 dans la numerotation des pages.

%%%%%%%%%%
% PREMIERE DE COUVERTURE
%%%%%%%%%%

\newlength{\plarg}
\setlength{\plarg}{8cm}
\newlength{\glarg}
\setlength{\glarg}{14cm}
\newlength{\Glarg}
\setlength{\Glarg}{15cm}

\begin{titlepage}
\thispagestyle{empty}
%\noindent
%\noindent
%\begin{minipage}{\glarg}
%{\large LOGO labo} \hfill {\large LOGO Ecole} 
%{\rule{\glarg}{1pt}} \vspace{-1cm}
%\begin{center}
%\begin{tabular}{p{13cm}r}
%\hspace{-2mm}\textbf{Nom Labo} & \textbf{LABO} \\
%\hspace{-2mm}Nom complet & Section \dots \\
%\hspace{-2mm}du laboratoire & 
%\end{tabular}
%\end{center}
%\end{minipage}

\vspace{-1cm}
\begin{minipage}{\glarg}
\vspace{-4.5cm}
\hspace{-1.5cm}{\color{gris}\large\bf N$^o$ d'ordre  : \hspace{9.5cm} \bf ANN\'EE 2017}
\end{minipage}
\vspace{-2cm}
\begin{figure}[htp]
\center
\hspace{-1cm}\includegraphics[angle=0,width=15cm]{logo.png}
\end{figure}

\begin{center}
\begin{minipage}{\glarg}
\vspace{0.5cm}
\centering{\Large\bfseries TH\`ESE / UNIVERSIT\'E DE RENNES 1}\\ \vspace{0mm}\emph{\Large sous le sceau de l'Universit\'e Bretagne Loire}\\ \vspace{0.5cm}
{\Large pour le grade de}\\ \vspace{2mm}
{\Large\bf DOCTEUR DE L'UNIVERSIT\'E DE RENNES 1}\\ \vspace{0.4cm}
\emph{\Large Mention :  Traitement du Signal et T\'el\'ecommunications}\\ \vspace{2mm}
{\Large\bf \'Ecole doctorale MATISSE}\\ \vspace{0.3cm}
{\Large pr\'esent\'ee par} \\ \vspace{3mm}
{\Huge\bf Xuan-Chien LE}\\ \vspace{0.4cm}
{\Large pr\'epar\'ee \`a l'unit\'e de recherche UMR6074 IRISA\\
\hspace{-1cm}Institut de recherche en informatique et syst\`emes al\'eatoires - CAIRN\\
\hspace{-1cm}\'Ecole Nationale Sup\'erieure des Sciences Appliqu\'ees et de Technologie}\vspace{0.3cm}
\\
\hspace{-20mm}{\rule{\Glarg}{1pt}}\\
\vspace{8mm}

\begin{tabular}{p{7cm}p{10cm}}
\begin{minipage}{\plarg}
%\vspace{-4cm}

%Global power management system for self-powered autonomous wireless sensor node

\hspace{-1.8cm}{\huge\bf Improving}\vspace{5mm}

\hspace{-1.8cm}{\huge\bf Performance of}\vspace{5mm}

\hspace{-1.8cm}{\huge\bf Non-Intrusive }\vspace{5mm}

\hspace{-1.8cm}{\huge\bf Load Monitoring}\vspace{5mm}

\hspace{-1.8cm}{\huge\bf with  Low-Cost}\vspace{5mm}

\hspace{-1.8cm}{\huge\bf Sensor Networks}\vspace{5mm}

\end{minipage}
&
\begin{minipage}{\plarg}
{\large\bf Th\`ese soutenue \`a Lannion \vspace{0mm}\newline}
{\large\bf le  \vspace{2mm}\newline}
{\large devant le jury compos\'e de : \vspace{2mm}\newline}
{\large\bf DIOURIS Jean-Fran\c cois \vspace{0mm}\newline}
{Professeur \`a l'Universit\'e de Nantes \!/\! rapporteur\vspace{1mm}\newline}
{\large\bf CLAVIER Laurent \vspace{0mm}\newline}
{Professeur \`a l'Institut Mines Telecom \!/\! rapporteur\vspace{1mm}\newline}
{\large\bf BACHA Seddik \vspace{0mm}\newline}
{Professeur \`a l'Universit\'e de Grenoble \!/\! examinateur\vspace{1mm}\newline}
{\large\bf MENGA David \vspace{0mm}\newline}
{Ing\'enieur Chercheur \`a EDF Lab \!/\! examinateur\vspace{1mm}\newline}
{\large\bf LEPRETTRE Beno\^it \vspace{0mm}\newline}
{Ing\'enieur Chercheur \`a Schneider Electric \!/\! examinateur\vspace{1mm}\newline}
{\large\bf LANGLAIS Charlotte \vspace{0mm}\newline}
{Ma\^itre de Conf\'erences \`a Telecom Bretagne \!/\! examinateur\newline}
{\large\bf SENTIEYS Olivier \vspace{0mm}\newline}
{Professeur \`a l'Universit\'e de Rennes 1 \!/\!  directeur de th\`ese\vspace{1mm}\newline}
{\large\bf VRIGNEAU Baptiste \vspace{0mm}\newline}
{Ma\^itre de Conf\'erences \`a l'Universit\'e de Rennes 1 \!/\! co-directeur de th\`ese\newline}
\end{minipage}
\end{tabular}

\end{minipage}
\end{center}
\end{titlepage}

\end{document}