\section{Conclusions}
In this chapter, we introduce a new load monitoring system, which uses a WSN to detect and provide the operating probability of some devices to help the NILM algorithms in improving their performance. This system is called sensor-aided non-intrusive load monitoring (SmartSense). Although deploying a sensors network, SmartSense is less intrusive than other intrusive load monitoring systems such as ViridiScope~\cite{Kim09Ubicomp}, because only a subset of all devices is selected to attach with the sensors and the identification is still based on an NILM algorithm.
The operation probability is used as a feature and combined with the traditional features of NILM by a regularization parameter to create a new one. In this chapter, two approaches are proposed:
\begin{itemize}
\item Knapsack approach: the $l1$-norm minimization problem in \cite{Hart92} is modeled as a knapsack problem and solved by two proposed algorithms including compositional Pareto-algebraic heuristic, which is based on a recursive relation, and dynamic programming, based the the construction of profit tables.
\item Edge detector approach: an edge detector is applied to detect the rising edge and falling edge of an activation on the aggregate power signal and to compare them with the existing ones in the library. The rising edge and falling edge can be directly used as a feature (as in edge detection algorithm) or combined with other active power values between them (as in dynamic time warping algorithm).
\end{itemize}

By simulating all four proposed algorithms with four dataset including UK-DALE~5, REDD~1, REDD~2 and our Athemium data, the performance of the load detection can be significantly improved with some monitored devices in comparison with traditional NILM systems. To obtain a good improvement, the devices selected to be monitored need to satisfy some criteria. In the knapsack approach, selected devices are the one with high using rate and confusion on the power demand with other devices. Meanwhile, the edge detector approach needs to monitor the devices more frequently switched on/off and ambiguous with the others on the edges height.
As a consequence, only several devices in home or building are selected, satisfying these requirements and allowing to significantly improve the overall performance when monitored. The others are less effective and need to be ignored when deploying the monitoring sensor network to reduce the cost. 

Besides, comparing the performance of the proposed approaches shows that the CPH and DP algorithms outperform the ED and DTW ones with more than two monitored devices in each dataset, although in normal NILM system, they are less effective. This result comes from the fact that the performance of the ED and DTW algorithms also depends on the edge detector capacity.

However, the algorithms of the knapsack approach show a high complexity presented by the large amount of multiplications and additions. To reduce the computational complexity, we can drive some parameters such as the number of Pareto points retained after each iteration in CPH or the scale factor in DP. Nevertheless, it also causes the attenuation of the performance. 